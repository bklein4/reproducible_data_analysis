% Options for packages loaded elsewhere
\PassOptionsToPackage{unicode}{hyperref}
\PassOptionsToPackage{hyphens}{url}
%
\documentclass[
]{article}
\usepackage{lmodern}
\usepackage{amssymb,amsmath}
\usepackage{ifxetex,ifluatex}
\ifnum 0\ifxetex 1\fi\ifluatex 1\fi=0 % if pdftex
  \usepackage[T1]{fontenc}
  \usepackage[utf8]{inputenc}
  \usepackage{textcomp} % provide euro and other symbols
\else % if luatex or xetex
  \usepackage{unicode-math}
  \defaultfontfeatures{Scale=MatchLowercase}
  \defaultfontfeatures[\rmfamily]{Ligatures=TeX,Scale=1}
\fi
% Use upquote if available, for straight quotes in verbatim environments
\IfFileExists{upquote.sty}{\usepackage{upquote}}{}
\IfFileExists{microtype.sty}{% use microtype if available
  \usepackage[]{microtype}
  \UseMicrotypeSet[protrusion]{basicmath} % disable protrusion for tt fonts
}{}
\makeatletter
\@ifundefined{KOMAClassName}{% if non-KOMA class
  \IfFileExists{parskip.sty}{%
    \usepackage{parskip}
  }{% else
    \setlength{\parindent}{0pt}
    \setlength{\parskip}{6pt plus 2pt minus 1pt}}
}{% if KOMA class
  \KOMAoptions{parskip=half}}
\makeatother
\usepackage{xcolor}
\IfFileExists{xurl.sty}{\usepackage{xurl}}{} % add URL line breaks if available
\IfFileExists{bookmark.sty}{\usepackage{bookmark}}{\usepackage{hyperref}}
\hypersetup{
  hidelinks,
  pdfcreator={LaTeX via pandoc}}
\urlstyle{same} % disable monospaced font for URLs
\usepackage[margin=1in]{geometry}
\usepackage{graphicx,grffile}
\makeatletter
\def\maxwidth{\ifdim\Gin@nat@width>\linewidth\linewidth\else\Gin@nat@width\fi}
\def\maxheight{\ifdim\Gin@nat@height>\textheight\textheight\else\Gin@nat@height\fi}
\makeatother
% Scale images if necessary, so that they will not overflow the page
% margins by default, and it is still possible to overwrite the defaults
% using explicit options in \includegraphics[width, height, ...]{}
\setkeys{Gin}{width=\maxwidth,height=\maxheight,keepaspectratio}
% Set default figure placement to htbp
\makeatletter
\def\fps@figure{htbp}
\makeatother
\setlength{\emergencystretch}{3em} % prevent overfull lines
\providecommand{\tightlist}{%
  \setlength{\itemsep}{0pt}\setlength{\parskip}{0pt}}
\setcounter{secnumdepth}{-\maxdimen} % remove section numbering

\author{}
\date{\vspace{-2.5em}}

\begin{document}

title: ``hw\_9\_Final\_Project\_Description'' author: ``Barbara Klein''
date: ``10/25/2020'' output: github\_document

\hypertarget{reproducible-data-analysis-final-project}{%
\section{\texorpdfstring{\textbf{Reproducible Data Analysis Final
Project}}{Reproducible Data Analysis Final Project}}\label{reproducible-data-analysis-final-project}}

\hypertarget{a-long-time-ago-in-a-body-of-water-far-far-away}{%
\subsection{A long time ago, In a Body of Water Far Far
Away\ldots{}}\label{a-long-time-ago-in-a-body-of-water-far-far-away}}

Great, now that you think this is a Star Wars parody, I will begin with
a brief introduction.

It all started in the summer of 86'\ldots{} Ok ok Hoops McCann isn't
telling the story either, so we'll start for real this time. Back in
2004, there was a group of scientists who set out to gather information
on both eukaryotic and prokaryotic microbial communities. Their
particular location involved penguins, ice bergs, and traveling through
other bone chilling bodies of water; I'm talking about the Ross Sea,
Antarctica. These scientists were on a mission to extrapolate a
relationship between UV light and microbial community composition at
various depths within the Southern Hemisphere Summer. For a long-term
project during my undergrad years, I had graciously received the
cruise's master spreadsheet to sort through and organize. Specific items
on this spreadsheet include station ID, latitude, longitude,
temperature, depth, and many other measurements. Knowing that oceanic
research cruises are driven by sleep deprived scientists, and having
just been handed oodles of spreadsheets and archived data to organize, I
came to find that certain measurements or pieces of data were missing or
unrecoverable. Being that this data may be lost at sea forever, here we
set out to fill in those missing data points with the help of the
almighty statistical powers that be, in an R Studio script.

During the research cruise, the ships log tracked the latitude and
longitude of each sampling site or station. These station locations hold
various measurements taken at depths ranging from the surface (0
meters), all the way down to 250 meters. The measurements that were
taken consisted of chlorophyll a, bacterial direct counts, leucine
incorporation, bacterial production, nitrate and nitrite concentration,
phosphate concentration, DSI, ammonium concentration, salinity, and
temperature. While some of the numerical measurements were extrapolated
from the ships' log, the biogeochemical measurements were gathered from
either preserved DNA filters, or water collection samples taken via CTD
Rosette. Some of these depths didn't have certain measurements recorded
and therefore couldn't piece together a full and complete story.

These stations correspond to specific latitude/longitudinal coordinates
which will be deemed x\_i. This indicates the literal point in space of
a single station (site) at which sampling occured. Of course we will
take depth into account, which will be y\_i. Within my dataset, there
are plenty of missing values at each depth. Using the ``Kriging Model'',
we will be predicting missing points using geostatistics, and creating a
map, and annotating an R Markdown file enough for someone completely new
(say an Undergrad) to R, to recreate a graph similar to ours.

Lovell will be creating the map, annotating the commands and what their
purpose is, and fitting this map to be universally used. I (Barb) will
be laying out the statistical modeling characters and equations,
annotating those characters, and also making them in such a way as to be
universally used. We would like to emphasize, there will be a lot of
annotating throughout the Markdown file in order to thoroughly explain
each detail in its' entirety so this Markdown file can truly be used at
the beginner level.

\end{document}
